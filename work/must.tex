\documentclass[uplatex, oneside]{jsbook}

\usepackage{setspace}

\usepackage[dvipdfmx]{hyperref}
\usepackage{pxjahyper}
\hypersetup{
    colorlinks=true,
    linkcolor=blue,
    urlcolor=blue,
}

\title{タイトル}
\author{作者}

\setcounter{secnumdepth}{5}
\setcounter{tocdepth}{5}

\begin{document}

\maketitle
\tableofcontents

\chapter{はじめに}

\section{セクション}

\href{http://www.google.co.jp}{グーグル}
\footnote{脚注}

\newpage

あいうえお\footnote{特殊文字なし}

かきくけこ\footnote{特殊文字あり\#}

さしすせそ\footnote{特殊\$文字あり}

たちつてと\footnote{特殊\textbackslash 文字あり}

\newpage

\section{セクション1}

\section{セ\#ク\$ション2}

\subsection{サブセクション\textbackslash}

\subsubsection{サブサブ\%セクション}

\paragraph{段落の名前}

\subparagraph{小段落の名前}

\newpage

\href{http://www.google.co.jp}{Google}

\href{http://localhost:3000/sample/page1}{タイトル1}

\href{http://localhost:3000/sample/page2}{title\_2}

\href{http://localhost:3000/sample/page3}{タイトル\$3\$}

\chapter{}

\newpage

\begin{verbatim}
  \href{https://www.google.co.jp/}{Google}\footnote{\url{https://www.google.co.jp/}}
\end{verbatim}

\href{https://www.google.co.jp/}{Google}\footnote{\url{https://www.google.co.jp/}}

\end{document}